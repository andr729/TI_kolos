\documentclass[final,12pt]{article}
\usepackage{amsmath}
\usepackage{amssymb}
\usepackage{latexsym}
\usepackage{mathtools}
\usepackage[margin=1in]{geometry}
\usepackage[polish,english]{babel}
\usepackage[utf8]{inputenc}
\usepackage[T1]{fontenc}

\usepackage{url}
\usepackage{xspace}
\usepackage[pdftex]{graphicx}
\usepackage[pdftex]{color}

\usepackage{adjustbox}


% \definecolor{azure}{rgb}{0.0, 0.5, 0.0}
% \definecolor{processblue}{cmyk}{0.96,0,0,0}

\DeclarePairedDelimiter\abs{\lvert}{\rvert}%
\DeclarePairedDelimiter\norm{\lVert}{\rVert}%
\DeclarePairedDelimiter\set{\{}{\}}%
\DeclarePairedDelimiter\tuple{\langle}{\rangle}%

% Swap the definition of \abs* and \norm*, so that \abs
% and \norm resizes the size of the brackets, and the 
% starred version does not.
\makeatletter
\let\oldabs\abs
\def\abs{\@ifstar{\oldabs}{\oldabs*}}
%
\let\oldnorm\norm
\def\norm{\@ifstar{\oldnorm}{\oldnorm*}}
%
\let\oldset\set
\def\set{\@ifstar{\oldset}{\oldset*}}
%
\let\oldtuple\tuple
\def\tuple{\@ifstar{\oldtuple}{\oldtuple*}}
%
\makeatother
%



\begin{document}
\setlength{\parindent}{0pt}
\noindent

{\bf Teoria informacji 2023/24. Kolokwium -- Zadanie 3}

Andrzej Radzimiński, ar429586
\\

\ \\

\newcommand{\ot}{\frac{1}{3}}
\newcommand{\twt}{\frac{2}{3}}

\renewcommand*{\arraystretch}{1.5}

Dane są kanały, reprezentowane macierzami:
$$
A =
\begin{pmatrix}
	0   & \ot & \ot & \ot \\
	\ot & 0   & \ot & \ot \\
	\ot & \ot & 0   & \ot \\
	\ot & \ot & \ot & 0   \\
\end{pmatrix}
\ \ \ \ \ \ \ \ \ \  
B = 
\begin{pmatrix}
	0   & \ot & \ot & \ot \\
	0   & 0   & \ot & \twt \\
	\ot & \twt& 0   & 0  \\
	\ot & \ot & \ot & 0   \\
\end{pmatrix}
$$

Pokażę, że kanał $B$ posiada większą przepustowość.


\end{document}