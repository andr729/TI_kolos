\documentclass[final,12pt]{article}
\usepackage{amsmath}
\usepackage{amssymb}
\usepackage{latexsym}
\usepackage{mathtools}
\usepackage[margin=1in]{geometry}
\usepackage[polish,english]{babel}
\usepackage[utf8]{inputenc}
\usepackage[T1]{fontenc}

\usepackage{url}
\usepackage{xspace}
\usepackage[pdftex]{graphicx}
\usepackage[pdftex]{color}

\usepackage{adjustbox}


% \definecolor{azure}{rgb}{0.0, 0.5, 0.0}
% \definecolor{processblue}{cmyk}{0.96,0,0,0}

\DeclarePairedDelimiter\abs{\lvert}{\rvert}%
\DeclarePairedDelimiter\norm{\lVert}{\rVert}%
\DeclarePairedDelimiter\set{\{}{\}}%
\DeclarePairedDelimiter\tuple{\langle}{\rangle}%

% Swap the definition of \abs* and \norm*, so that \abs
% and \norm resizes the size of the brackets, and the 
% starred version does not.
\makeatletter
\let\oldabs\abs
\def\abs{\@ifstar{\oldabs}{\oldabs*}}
%
\let\oldnorm\norm
\def\norm{\@ifstar{\oldnorm}{\oldnorm*}}
%
\let\oldset\set
\def\set{\@ifstar{\oldset}{\oldset*}}
%
\let\oldtuple\tuple
\def\tuple{\@ifstar{\oldtuple}{\oldtuple*}}
%
\makeatother
%



\begin{document}
\setlength{\parindent}{0pt}
\noindent

{\bf Teoria informacji 2023/24. Kolokwium -- Zadanie 3}

Andrzej Radzimiński, ar429586
\\

\ \\

\newcommand{\ot}{\frac{1}{3}}
\newcommand{\twt}{\frac{2}{3}}

\renewcommand*{\arraystretch}{1.5}

Dane są kanały, reprezentowane macierzami:
$$
K \equiv
\begin{pmatrix}
	0   & \ot & \ot & \ot \\
	\ot & 0   & \ot & \ot \\
	\ot & \ot & 0   & \ot \\
	\ot & \ot & \ot & 0   \\
\end{pmatrix}
\ \ \ \ \ \ \ \ \ \  
M \equiv 
\begin{pmatrix}
	0   & \ot & \ot & \ot \\
	0   & 0   & \ot & \twt \\
	\ot & \twt& 0   & 0  \\
	\ot & \ot & \ot & 0   \\
\end{pmatrix}
$$

Pokażę, że kanał $M$ posiada większą przepustowość.

Oznaczmy wejście kanału jako $A$ i wyjście kanału jako $B$.\\

Policzmy najpierw przepustowość kanału $K$:
$$
	C_K = \max_{A} I(A; B).
$$

Oznaczmy prawdopodobieństwo kolejnych "wejść" kanału jako
$p_0, p_1, p_2, p_3$.

Teraz:
$$
	I(A; B) = H(B) - H(B | A).
$$
\begin{multline*}
	H(B) =
	\sum_{i} \frac{(1-p_{i})}{3} \cdot \log\left(\frac{3}{(1-p_{i})} \right) = \\
	\sum_{i} \frac{(1-p_{i})}{3} \cdot \left(\log(3) - \log(1-p_{i}) \right) = \\
	\sum_{i} \frac{(1-p_{i})}{3}\log(3) - \sum_{i} \frac{(1-p_{i})}{3}\log(1-p_{i}) = \\
	\frac{\log(3)}{3} \sum_{i} (1-p_{i}) - \frac{1}{3}\sum_{i} (1-p_{i})\log(1-p_{i}) = \\
	\frac{\log(3)}{3} \cdot 3 - \frac{1}{3}\sum_{i} (1-p_{i})\log(1-p_{i}) = \\
	\log(3) - \frac{1}{3}\sum_{i} (1-p_{i})\log(1-p_{i})
\end{multline*}

\begin{multline*}
	H(B | A) = 3 \cdot \frac{1}{3} \log{\frac{3}{1}} = \log{3}. \\
\end{multline*}

Zatem:
\begin{multline*}
	I(A; B) = H(B) - H(B | A) = \log(3) - \frac{1}{3}\sum_{i} (1-p_{i})\log(1-p_{i}) - \log(3) = \\
	- \frac{1}{3}\sum_{i} (1-p_{i})\log(1-p_{i}) = \\
	\frac{1}{3}\sum_{i} (1-p_{i})\log\left(\frac{1}{1-p_{i}}\right).
\end{multline*}

Funkcja $(1-x)\cdot \log(1/(1-x))$ przyjmuje na przedziale $[0, 1]$ maksimum w punkcje $x = 1 - 1/e$, które wynosi:
$$
q
$$

\end{document}