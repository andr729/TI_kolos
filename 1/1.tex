\documentclass[final,12pt]{article}
\usepackage{amsmath}
\usepackage{amssymb}
\usepackage{latexsym}
\usepackage{mathtools}
\usepackage[margin=1in]{geometry}
\usepackage[polish,english]{babel}
\usepackage[utf8]{inputenc}
\usepackage[T1]{fontenc}

\usepackage{url}
\usepackage{xspace}
\usepackage[pdftex]{graphicx}
\usepackage[pdftex]{color}

\usepackage{adjustbox}


% \definecolor{azure}{rgb}{0.0, 0.5, 0.0}
% \definecolor{processblue}{cmyk}{0.96,0,0,0}

\DeclarePairedDelimiter\abs{\lvert}{\rvert}%
\DeclarePairedDelimiter\norm{\lVert}{\rVert}%
\DeclarePairedDelimiter\set{\{}{\}}%
\DeclarePairedDelimiter\tuple{\langle}{\rangle}%

% Swap the definition of \abs* and \norm*, so that \abs
% and \norm resizes the size of the brackets, and the 
% starred version does not.
\makeatletter
\let\oldabs\abs
\def\abs{\@ifstar{\oldabs}{\oldabs*}}
%
\let\oldnorm\norm
\def\norm{\@ifstar{\oldnorm}{\oldnorm*}}
%
\let\oldset\set
\def\set{\@ifstar{\oldset}{\oldset*}}
%
\let\oldtuple\tuple
\def\tuple{\@ifstar{\oldtuple}{\oldtuple*}}
%
\makeatother
%



\begin{document}
\setlength{\parindent}{0pt}
\noindent

{\bf Teoria informacji 2023/24. Kolokwium -- Zadanie 1}

Andrzej Radzimiński, ar429586
\\

\ \\

Rozpatrzmy przypadek dla $n$ monet ($n \geq 7$).

Niech $M$ oznacza zmienną losową reprezentującą to, która moneta jest lżejsza:
$$
	M \in \set{1, 2, 3, 4, \ldots, n}.
$$

Niech $W$ oznacza zmienną losową reprezentującą to, czy waga przechyliła się w lewo, w prawo czy pozostała wyrównana:
$$
	W \in \set{L, R, E}.
$$

Bez straty ogólności, możemy przyjąć że ważone są odpowiednio monety $1,2,3 \text{ vs } 3,4,6$ lub $1,2 \text{ vs } 3,4$.

Interesuje nas teraz informacja, o tym w którym przypadku większe jest:
$$
	I(M, W) = H(W) + H(M) - H(W,M).
$$
Najpierw zauważmy, że $W$ jest jednoznacznie zdeterminowana przez $M$, zatem:
$$
	H(W,M) = H(M),
$$
z czego wynika:
$$
	I(M, W) = H(W) + H(M) - H(W,M) = H(W).
$$

Wiadomo, że:
$$
	H(W) = p(L) \cdot \log(1/p(L)) + p(R) \cdot \log(1/p(R)) + p(E) \cdot \log(1/p(E)).
$$
Dodatkowo, możemy zauważyć, że $p(L) = P(R)$, co dodatkowo uprasza wzór, do:
$$
	H(W) = 2 \cdot p(L) \cdot \log(1/p(L)) + p(E) \cdot \log(1/p(E)).
$$

\subsection*{Ważenie trzech v trzech monet}

W tym przypadku $P(L) = 3/n$, $P(E) = (n-6)/n$, zatem:
\begin{align*}
	H(W) = 2 \cdot \frac{3}{n} \cdot \log(\frac{n}{3}) + \frac{n-6}{n} \cdot \log(\frac{n}{n-6}) =\\
	\frac{6}{n} \cdot (\log(n) - \log(3)) + \frac{n-6}{n}\cdot (\log(n)-\log(n-6)) = \\
	\log(n) - \frac{6\log(3) + (n-6)\log(n-6)}{n}
	% = \\
	% -\frac{6\log(3) - 6\log(n-6)}{n} = \\
	% \frac{6\log(n-6) - 6\log(3)}{n} = \\
	% \frac{6}{n}(\log(n-6) - \log(3))
\end{align*}

Oznaczmy wynik jako:
$$
	I_{33}(n) = \log(n) - \frac{6\log(3) + (n-6)\log(n-6)}{n}.
$$

\subsection*{Ważenie dwóch v dwóch monet}

W tym przypadku $P(L) = 2/n$, $P(E) = (n-4)/n$, zatem:
\begin{align}
	H(W) = 2 \cdot \frac{2}{n} \cdot \log(\frac{n}{2}) + \frac{n-4}{n} \cdot \log(\frac{n}{n-4}) =\\
	\frac{4}{n} \cdot (\log(n) - \log(2)) + \frac{n-4}{n}\cdot(\log(n) - \log)(n-4) = \\
	\log(n) - \frac{4 + (n-4)\log(n-4)}{n}
	% = \\
	% -\frac{4 - 4\log(n-4)}{n} = \\
	% \frac{4\log(n-4) - 4}{n} = \\
	% \frac{4}{n}(\log(n-4) - 1)
\end{align}

Oznaczmy wynik jako:
$$
	I_{22}(n) = \log(n) - \frac{4 + (n-4)\log(n-4)}{n}.
$$

\subsection*{$7$ monet}
Dla siedmiu monet:
$$
	I_{33}(7) \approx 1.4488156357251847,
$$
$$
	I_{22}(7) \approx 1.556656707462823,
$$
zatem więcej informacji otrzymujemy ważąc dwie przeciw dwóm monetom.

\subsection*{Przypadek ogólny}
Rozpatrzmy równanie:
$$
	I_{33}(n) - I_{22}(n) > 0.
$$
Przekształcając równoważnie otrzymujemy:
$$
\log(n) - \frac{6\log(3) + (n-6)\log(n-6)}{n} - \log(n) + \frac{4 + (n-4)\log(n-4)}{n} > 0
$$
$$
 - \frac{6\log(3) + (n-6)\log(n-6)}{n} + \frac{4 + (n-4)\log(n-4)}{n} > 0
$$
$$
 -(6\log(3) + (n-6)\log(n-6)) + (4 + (n-4)\log(n-4)) > 0
$$
$$
(4 + (n-4)\log(n-4))-(6\log(3) + (n-6)\log(n-6)) > 0
$$
$$
4 + (n-4)\log(n-4) - 6\log(3) - (n-6)\log(n-6) > 0
$$
$$
(4 - 6\log(3)) + (n-4)\log(n-4) - (n-6)\log(n-6) > 0
$$


\end{document}